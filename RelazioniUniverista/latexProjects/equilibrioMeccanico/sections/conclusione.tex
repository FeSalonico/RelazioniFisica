\section*{Conclusione}

Dalle misurazioni effettuate si osserva che, in accordo con i principi della dinamica, la somma delle forze agenti sul corpo non è apprezzabile dagli strumenti di misura utilizzati.\\
In particolare si osserva che la somma delle forze in direzione $x$ agenti sul corpo è mediamente pari a $-0.06 N \pm 0.09N$, mentre in direzione $y$ è mediamente pari a $0.2 N \pm 0.3N$.\\ 
Si ipotizza che la differenza tra la precisione delle due misure sia dovuta al fatto che nella sola direzione $y$ si confrontano forze di 
diversa natura (forza peso e forze elastiche).\\
Da questa osservazione si può dedurre che la forza elastica analizzata non abbia una dipendenza perfettamente lineare con l'allungamento del copro elastico e che questo fattore è meno evidente quando si 
confrontano due forze elastiche date da allungamenti paragonabili di elastici simili.\\


