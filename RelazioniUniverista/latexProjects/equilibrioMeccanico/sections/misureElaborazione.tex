\section*{Misure ed elaborazione dati}
\everymath{\displaystyle}
$\theta_A = 43^\circ$\\ 
$\theta_B = 76^\circ$\\ 
$l_{A_0} = l_{B_0} = 105.5\;mm \pm 0.5\;mm = l_0$\\ 
$k_A = k_B = 70 Nm^{-1} \pm 5 Nm^{-1} = k$
\begin{table}[h!]
    
    \hspace{5mm} 
    \begin{tabular}{|c|c|c|}
        \hline
        $i$ & $d_i$ (diametro) & $c_i$ (circonferenza)\\
        \hline

        1  & $37.0 \;mm \pm 0.5 \;mm$ & $116 \;mm \pm 1 \;mm$\\ 
        2  & $54.0 \;mm \pm 0.5 \;mm$ & $171 \;mm \pm 1 \;mm$\\ 
        3  & $61.0 \;mm \pm 0.5 \;mm$ & $192 \;mm \pm 1 \;mm$\\ 
        4  & $70.5 \;mm \pm 0.5 \;mm$ & $218 \;mm \pm 1 \;mm$\\ 
        5  & $85.5 \;mm \pm 0.5 \;mm$ & $271 \;mm \pm 1 \;mm$\\ 
        6  & $89.5 \;mm \pm 0.5 \;mm$ & $279 \;mm \pm 1 \;mm$\\ 
        7  & $104.5 \;mm \pm 0.5 \;mm$ & $325 \;mm \pm 1 \;mm$\\ 
        8  & $134.0 \;mm \pm 0.5 \;mm$ & $408 \;mm \pm 1 \;mm$\\ 
        9  & $149.5 \;mm \pm 0.5 \;mm$ & $466 \;mm \pm 1 \;mm$\\ 
        10 & $151.0 \;mm \pm 0.5 \;mm$ & $476 \;mm \pm 1 \;mm$\\ 
        11 & $167.5 \;mm \pm 0.5 \;mm$ & $531 \;mm \pm 1 \;mm$\\ 
        12 & $198.5 \;mm \pm 0.5 \;mm$ & $625 \;mm \pm 1 \;mm$\\ 
        13 & $211.0 \;mm \pm 0.5 \;mm$ & $660 \;mm \pm 1 \;mm$\\ 
        14 & $220.0 \;mm \pm 0.5 \;mm$ & $688 \;mm \pm 1 \;mm$\\ 
        15 & $293.5 \;mm \pm 0.5 \;mm$ & $911 \;mm \pm 1 \;mm$\\ 


        \hline
    \end{tabular}
    \caption{Misure rilevate}
    \label{tabellaDati}
\end{table}

\snls 
Si osserva che il corpo è inizialmente in quiete e rimane in quiete.\\ 
Dunque, per il secondo principio della dinamica, la somma delle forze agenti sul corpo è nulla.\\ 
In particolare:\\ \\
$
\begin{cases}
    x: \qq \F_{A_x} + \F_{B_x} = 0\\
    y: \qq \F_{A_y} + \F_{B_y} + \vb{P} = 0
\end{cases}
\\ 
\phantom{A}
\\
\begin{cases}
    x: \qq  F_{B_x} - F_{A_x} = 0\\
    y: \qq F_{A_y} + F_{B_y}  -m_ig = 0
\end{cases} 
\\
\phantom{A}
\\
\begin{cases}
    x: \qq  F_{B} \cos \theta_B - F_{A} \cos \theta_A = 0\\
    y: \qq F_{A}\sin \theta_A  + F_{B}\sin \theta_B -m_ig  = 0
\end{cases} 
\\ 
\phantom{A}
\\
\begin{cases}
    x: \qq  k (l_{B_i} - l_0) \cos \theta_B - k (l_{A_i} - l_0) \cos \theta_A = 0\\
    y: \qq k (l_{A_i} - l_0)\sin \theta_A  + k (l_{B_i} - l_0)\sin \theta_B -m_ig  = 0
\end{cases} 
$ 
\begin{table}[h!]
    
    \hspace{5mm} 
    \begin{tabular}{|c|c|c|}
        \hline
        $i$ & $\F_{A_x} + \F_{B_x}$& $\F_{A_y} + \F_{B_y} + \vb{P}$\\
        \hline
        1 & $(0.48 \pm 0.05)\; N$  & $(112.5\pm 0.5) \;N$\\
        2 & $(0.53 \pm 0.05)\; N$  & $(110.5\pm 0.5) \;N$\\
        3 & $(0.56 \pm 0.05)\; N$  & $(107.5\pm 0.5) \;N$\\
        4 & $(0.56 \pm 0.06)\; N$  & $(115.5\pm 0.5) \;N$\\
       


        \hline
    \end{tabular}
    \caption{Risultati}
    \label{tabellaRis}
\end{table}
